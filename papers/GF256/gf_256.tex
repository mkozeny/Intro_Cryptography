\title{GF 256} \author{
        Martin Kozeny\\
        MATH 4530: Intro Cryptography\\
        Spring 2011
        University of New Orleans
}
\date{\today}




\documentclass[5pt]{article}
\usepackage{graphicx}
\usepackage{amssymb}
\usepackage{amsmath}
\usepackage{qtree}
\usepackage{multicol}
%\usepackage{chemarrow}
\usepackage[utf8]{inputenc}


\setlength{\hoffset}{-2.3cm} 
\setlength{\voffset}{-3cm}
\setlength{\textheight}{24.0cm} 
\setlength{\textwidth}{16cm}


\begin{document}


\maketitle

\section{Adding}
For adding, I tried to evaluate these examples with shown results in
table \ref{tab:adding}.

\begin{table}[ht]
  \centering
  \caption{Adding in GF 256}
  \begin{tabular}{|l|l|l|l|}
  \hline
  Left operand & Right operand & Expected result & Result\\
  \hline
  \hline
  
  A1 & D7 & 76 & 76 \\
  \hline
57 & 83 & D4 & D4\\
\hline
4B & C8 & 83 & 83\\
\hline
B2 & E5 & 57 & 57 \\
\hline
F1 & 10 &  E1 & E1\\
\hline 	
D7 & 12 & C5 & C5 \\
\hline 	
07& B9 & BE &	BE\\
\hline

  \end{tabular}
  \label{tab:adding}
\end{table}


\section{Multiplying}
For multiplying, I tried to evaluate these examples with shown results in
table \ref{tab:multiplying}.

\begin{table}[ht]
  \centering
  \caption{Multiplying in GF 256}
  \begin{tabular}{|l|l|l|l|}
  \hline
  Left operand & Right operand & Expected result & Result\\
  \hline
  \hline
  
  A1 & D7 & D0 & D0 \\
  \hline
57 & 83 & C1 & C1\\
\hline
4B & C8 & 89 & 89\\
\hline
B2 & E5 & 76 & 76 \\
\hline
F1 & 10 &  89 & 89\\
\hline 	
D7 & 12 & 6A & 6A \\
\hline 	
07& B9 & 2 &	2\\
\hline

  \end{tabular}
  \label{tab:multiplying}
\end{table}


\newpage
\section{Finding inverses}
For finding inverses, I tried to evaluate these examples with shown results in
table \ref{tab:inverses}.

\begin{table}[ht]
  \centering
  \caption{Finding inverses in GF 256}
  \begin{tabular}{|l|l|l|}
  \hline
  Operand &  Expected result & Result\\
  \hline
  \hline
  
  7E & 81 & 81 \\
  \hline
4B & 4B & 13\\
\hline
C6 & E4 & E4\\
\hline
B2 & 1F & 1F \\
\hline
F1 & 23 & 23\\
\hline 	
D7 & EA & EA \\
\hline 	
07& D1 &	D1\\
\hline

  \end{tabular}
  \label{tab:inverses}
\end{table}


\section{Conclusion}
In conclusion it appears from provided test, that implemented algorithm works
correctly.


\end{document}
