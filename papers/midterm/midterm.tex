\title{Midterm} \author{
        Martin Kozeny, David Kalivoda\\
        CSCI 4501: Intro Cryptography\\
        Spring 2011
        University of New Orleans
}
\date{\today}




\documentclass[5pt]{article}
\usepackage{graphicx}
\usepackage{amssymb}
\usepackage{amsmath}
\usepackage{qtree}
\usepackage{multicol}
%\usepackage{chemarrow}
\usepackage[utf8]{inputenc}


\setlength{\hoffset}{-2.3cm} 
\setlength{\voffset}{-3cm}
\setlength{\textheight}{24.0cm} 
\setlength{\textwidth}{16cm}


\begin{document}


\maketitle

\begin{enumerate}
  \item Sketch out in pseudocode the 'memory efficient' (that is, a version
  that does not require it to remember all of the steps) version of the extended Euclidean
algorithm discussed in class.

\begin{verbatim}
func euclidean(var Zn, var n)

  //initialization of variables to start looping properly
  var prevA=0;
  var prevB=1;
  var prevprevA=1;
  var prevprevB=0;
  
  //loop computing extended euclidean algorithm, the result is in prevA and prevB
  //loop uses continous dividing Zn by n and shifting parameters from each round
  loop( (Zn%n) != 0){
   a = Zn/n;
   b = Zn%n;
   
   //we need this temp variables for swapping coeficients
   //from prevprev to prev (preparation for next round)
   tempA = prevA;
   tempB = prevB;
   
   //equations similar to those when we do in hand algorithm
   prevA = prevprevA - a*prevA;
   prevB = prevprevB - a*prevB;
   
   prevprevA = tempA;
   prevprevB = tempB;
   
   Zn = n;
   n  = b;    
  }//end of loop
  //inverse condition
  if (n!=1) Inverse does not exist;
  
  //in return container are the multiplicative constantes of eucledian
  //expansion
  //we usually need only the prevB constant
  return prevB; 
\end{verbatim}
\newpage
  \item Sketch out in pseudocode the algorithm which allows us to calculate
  $x^y \mod m$ for very large values of $x$, $y$, $m$.
\begin{verbatim}
func power(var x, var y, var Zn)
  y = toBinaryNumber(y);
  //function convertToExpression() coverts input binary number
  //such that instead of ones puts 'X' and into spaces between bits puts 'S'
  //e.g. 23 = 10101 in binary -> XSSXSSX
  var powerExpression = convertToExpression(y);
  //now we are going to do powering x to y
  var result = x;
  //this looping starts from most significant char in string of powerExpression
  loop through bits of powerExpression{
    if 'X'
      multiply result from previous loop by x mod Zn 
    else if 'S'
      square result from previous loop mod Zn
  }//end of loop
  //now x^y is in result
  return result
\end{verbatim}
Note: The most expensive operation in this function is squaring and moduling,
which we can compute manually.
  \item Do the following in
$GF (16) = ((Z/_2)[x])/(x^4 + x + 1)$


Fully simplify your answer.
\begin{description}
\item[(a)] Compute $(x^2 + 1)(x^3 + x^2 + 1)$.

$$(x^2 + 1)(x^3 + x^2 + 1) = x^5 + x^4 + x^3 + 1$$
$$(x^5 + x^4 + x^3 + 1)/(x^4 + x + 1) = x + 1,\ remainder\ x^3 + x^2$$
$$(x^2 + 1)(x^3 + x^2 + 1) \equiv x^3 + x^2\ in\ GF (16) = ((Z/_2)[x])/(x^4 +
x + 1)$$

\item[(b)] Find the inverse of $x^2 + x + 1$.

$$(x^4 + x + 1)/(x^2 + x + 1) = x^2 + x,\ remainder\ 1$$
$$(x^4 + x + 1) = (x^2 + x)(x^2 + x + 1) + 1$$
$$1 = (x^2 + x)(x^2 + x + 1) + 1(x^4 + x + 1)$$

Multiplicative inverse of $(x^2 + x + 1)$\  in\ $GF (16) = ((Z/_2)[x])/(x^4 + x
+ 1)$ is $(x^2 + x)$.
\end{description}
\item Let, $p = 5$, $q = 7$, and $e = 17$ for RSA encryption. Set $x = 13$.
Encrypt $x$ using RSA. Then find $d$ and decrypt back to the original message.

$$n = pq = 5\cdot7 = 35$$
$$\phi(35) = (5 - 1)(7 - 1) = 24$$
$$17 = (10001)_2 = XSSSSX$$
For computation $13^{17}$\ in\ $\mathbb{Z}_{35}$ we used algorithm shown in
second question.
$$13^{17} \equiv 13 \mod{35}$$
Encrypted message is $13$.
For decrypting we have to find inverse $d = e^{-1}$, which is $17^{-1}$\ in\
$\mathbb{Z}_{24}$.

$$24 = 1\cdot17 + 7$$
$$17 = 2\cdot7 + 3$$
$$7 = 2\cdot3 + 1$$\newline
$$7 = 24 - 1\cdot17$$
$$3 = 17 - 2\cdot24 + 2\cdot17 = 3\cdot17 - 2\cdot24$$
$$1 = 24 - 1\cdot17 - 6\cdot17 + 4\cdot24 = 5\cdot24 - 7\cdot17$$

$$d = -7 \equiv 17\ in\ \mathbb{Z}_{24}$$

Because $d = e$, encrypting is the same as decrypting. We calculate $13^{17}
\equiv 13$ in $\mathbb{Z}_{35}$ as shown before.
\item Let p be a prime. What is $(\prod_{i = 1}^{p-1} i) \mod{p}$? Fully justify
(i.e., prove) your answer.

$(\prod_{i = 1}^{p-1} i) \mod{p} = 1\cdot2\ldots(p - 1)$

We know, that all elements in $\mathbb{Z}_{p}$ have inverses. We can see, that
$1$ has inverse onto itself. We try to find, if any other elements of group
$\mathbb{Z}_{p}$ have the same quality.

$$(p - n)(p - n) \equiv 1 \mod p$$
\center{, where $n \in \{1, 2,\ldots,(p - 1)\}$}
 
$$n^2 - 2pn + p^2 - 1 \equiv 0 \mod p$$
$$(n - (p - 1))(n - (p + 1)) \equiv 0 \mod p$$
$$n_{1} = p + 1 \equiv 1 \mod p$$
$$n_{2} \equiv (p - 1) \mod p$$

We see that only $(p - n_1) \equiv (p - 1) \mod p$ and $(p - n_2) = (p - (p -1))
\equiv 1 \mod p$ have inverse onto itself. We might find \textbf{pairs of
inverses} for the rest elements in group. Each element multiplied by its inverse
is one, and therefore: $$(\prod_{i = 1}^{p-1} i) \mod{p} = 1\cdot\underbrace{2\cdot3\ldots(p -
2)}_{1}\cdot(p - 1) = (p - 1)$$

\end{enumerate}
\end{document}
