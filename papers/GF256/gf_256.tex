\title{GF 256} \author{
        Martin Kozeny\\
        MATH 4530: Intro Cryptography\\
        Spring 2011
        University of New Orleans
}
\date{\today}




\documentclass[5pt]{article}
\usepackage{graphicx}
\usepackage{amssymb}
\usepackage{amsmath}
\usepackage{qtree}
\usepackage{multicol}
%\usepackage{chemarrow}
\usepackage[utf8]{inputenc}


\setlength{\hoffset}{-2.3cm} 
\setlength{\voffset}{-3cm}
\setlength{\textheight}{24.0cm} 
\setlength{\textwidth}{16cm}


\begin{document}


\maketitle

\section{Implementation}
Details of implementation are written in comments in attached source code.
\textbf{Adding} is implemented as simple XOR, \textbf{multiplying} as
multiplying members of the first polynom with members of the second polynom
divided by modulo polynom (dividing was used instead of rewriting rules, beacause it is also
used in inverse finding). Algorithm of \textbf{inverse finding} is
the same as in previous homework - Extended Euclidean Algorithm - , but in
implementation are used polynoms instead of integers.

\section{Testing}
These tests were verified manually and also by \cite{gf256:web}.
\subsection{Adding}
For adding, I tried to evaluate these examples with shown results in
table \ref{tab:adding}.

\begin{table}[ht]
  \centering
  \caption{Adding in GF 256}
  \begin{tabular}{|l|l|l|l|}
  \hline
  Left operand & Right operand & Expected result & Result\\
  \hline
  \hline
  
  A1 & D7 & 76 & 76 \\
  \hline
57 & 83 & D4 & D4\\
\hline
4B & C8 & 83 & 83\\
\hline
B2 & E5 & 57 & 57 \\
\hline
F1 & 10 &  E1 & E1\\
\hline 	
D7 & 12 & C5 & C5 \\
\hline 	
07& B9 & BE &	BE\\
\hline 	
45& 2D & 68 &	68\\
\hline 	
CC& EE & 22 &	22\\
\hline
%neeext
AB & CD & 66 & 66 \\
\hline 	
3F& 7B & 44 &	44\\
\hline 	
62& 33 & 51 &	51\\
\hline 	
0E& 9D & 93 &	93\\
\hline


  \end{tabular}
  \label{tab:adding}
\end{table}

\newpage
\subsection{Multiplying}
For multiplying, I tried to evaluate these examples with shown results in
table \ref{tab:multiplying}.

\begin{table}[ht]
  \centering
  \caption{Multiplying in GF 256}
  \begin{tabular}{|l|l|l|l|}
  \hline
  Left operand & Right operand & Expected result & Result\\
  \hline
  \hline
  
  A1 & D7 & D0 & D0 \\
  \hline
57 & 83 & C1 & C1\\
\hline
4B & C8 & 89 & 89\\
\hline
B2 & E5 & 76 & 76 \\
\hline
F1 & 10 &  89 & 89\\
\hline 	
D7 & 12 & 6A & 6A \\
\hline 	
07& B9 & 2 &	2\\
\hline 	
45& 2D & 2C &	2C\\
\hline 	
CC& EE & A &	A\\
\hline
%neeext
AB & CD & BB & BB \\
\hline 	
3F& 7B & 87 &	87\\
\hline 	
62& 33 & 28 &	28\\
\hline 	
0E& 9D & E7 &	E7\\
\hline


  \end{tabular}
  \label{tab:multiplying}
\end{table}



\subsection{Finding inverses}
For finding inverses, I tried to evaluate these examples with shown results in
table \ref{tab:inverses}.

\begin{table}[ht]
  \centering
  \caption{Finding inverses in GF 256}
  \begin{tabular}{|l|l|l|}
  \hline
  Operand &  Expected result & Result\\
  \hline
  \hline
  
  7E & 81 & 81 \\
  \hline
4B & 13 & 13\\
\hline
C6 & E4 & E4\\
\hline
B2 & 1F & 1F \\
\hline
F1 & 23 & 23\\
\hline 	
D7 & EA & EA \\
\hline 	
07& D1 &	D1\\
\hline 	
45&  31 &	31\\
\hline 	
CC&  1B &	1B\\
\hline
%neeext
AB & 4A & 4A \\
\hline 	
3F& 19 & 19\\
\hline 	
62& AF &	AF\\
\hline 	
0E & E5 &	E5\\
\hline


  \end{tabular}
  \label{tab:inverses}
\end{table}


\section{Conclusion}
Provided tests show, that implemented algorithms work
correctly.

\begin{thebibliography}{1}

\bibitem{gf256:web}
GF256 calculator,\newline
\verb|http://computing.unn.ac.uk/staff/cgmb3/teaching/cryptography/GF256Applet/GF256Applet.html|.

\end{thebibliography}


\end{document}
